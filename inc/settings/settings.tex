%   =============================
%		Change document settings here 
%   =============================

% TITLEPAGE SETTINGS
% ------------------
\newcommand{\docAutor}					{Sebastian Kopf}								% name of the author of the book
\newcommand{\docTitel}					{Einf\"uhrung in die Funktionentheorie}			% title of the lecture/book
\newcommand{\docUntertitel}			{Sommersemester 2016}					% subtitle
\newcommand{\docDozent}					{Prof. Dr. N. V. Shcherbina}			% name of the professor
\newcommand{\docJahr}						{2016}										% publishing year
\newcommand{\docUniversitaet}		{Bergische Universit\"at Wuppertal}		% name of university
\newcommand{\docTitelZitat}			{Beweis ist relativ einfach. Haben kein Platz, also machen wir Platz.}
\newcommand{\docTitelZitatName}	{Prof. Dr. N. V. Shcherbina}				% titlepage quote author

% GENERAL SETTINGS
% ----------------
\newcommand{\useIndex}					{0} 											% set '1' if you want to include an index
\newcommand{\useThumbs}					{0}												% set '1' if you want to use chapter thumbs on the page side
\newcommand{\useRoman}					{0}												% set '1' if you want to have chapter numbering with roman numbers

\newcommand{\usrBCOR}						{0cm} 										% set binding correction offset here (space lost on the inner borders due to binding)
\newcommand{\usrmatter}					{0}												% set '1' to start numbering at actual content beginning
\newcommand{\usroptsqrt}				{0}												% set '1' to use alternate form for roots with a small closing line
\newcommand{\usrnscmd}[1]				{\mathbb{#1}}							% set e.g. '\textbf', '\mathbb' or '\mathds' for namespace macros
\newcommand{\sectioning}				{0}
				% set '1' to use section numbering (eg. 1.1.5)
\newcommand{\nulltesKapitelerzeugen}			{0}
\section[Zusammenh\"angende R\"aume, Gebiete in $ \C $]{Zusammenh\"angende R\"aume,\\Gebiete in $ \C $}
\begin{definition}
Sei $ (X, d_x) $ ein metrische Raum, $ A\subset X $ eine Menge. $ A $ ist \deftxt{zusammenh\"angend}$ \Leftrightarrow\nexists U_1,U_2 $ offen in $ X $, so dass:
\begin{enumerate}
\item $ U_1\cup U_2\supset A $
\item $ U_1\cap U_2=\emptyset $
\item $ U_1\cap A\neq\emptyset $, $ U_2\cap A\neq\emptyset $
\end{enumerate}
\end{definition}
\begin{beispiel*}
\begin{enumerate}
\item[]
\item $ \R=X $, $ d_x(x,y)=|x-y| $, $ A=\Q $: $ U_1=(-\infty,\sqrt{2}) $, $ U_2=(\sqrt{2},+\infty) $
\item $ \R=X $, $ d_x(x,y)=|x-y| $, $ A=[0,1] $.\\
Seien $ U_1,U_2 $ offene Mengen mit i)-iii), $ 0\in U_1 $, $ 1\in U_2 $, $ \frac{1}{2}\in U_1 \Rightarrow I_1=\left[\frac{1}{2},1\right]$, $ \frac{3}{4}\in U_2\Rightarrow I_2=\left[\frac{1}{2},\frac{3}{4}\right]$ $\Rightarrow\exists! x_0\in\bigcap_{n=1}^\infty I_n $ (Intervallschachtelungsprinzip). $ x_0 $ liegt also in $ U_1 $ oder $ U_2 $. $ U_1 $ ist offen, also existiert ein $ \e>0 $, so dass $ (x_0-\e,x_0+\e)\subset U_1 $, aber $ I_n\subset(x_0-\e,x_0+\e) $ f\"ur $ n $ gen\"ugend gro\ss $\lightning $ Also ist $ A $ zusammenh\"angend.
\end{enumerate} 
\end{beispiel*}
\begin{bemerkung*}
Sei $ (X,d_x) $ ein metrischer Raum, $ A,B\subset X $ zusammenh\"angend, $ A\cap B\neq\emptyset $. Dann ist $ A\cup B $ zusammenh\"angend.
\end{bemerkung*}
\begin{definition}
Sei $ (X,d_x) $ ein metrischer Raum, $ A\subset X $ eine Teilmenge. $ \forall x_0\in A $ definieren wir \[ K(x)\coloneqq\left\lbrace\bigcup_\alpha A_\alpha\mid x_0\in A_\alpha, A_\alpha\subset A\text{ zusammeh\"angend} \right\rbrace \]
$ K(x) $ hei\ss t \deftxt{Zusammenhangskomponente des Punktes $ x $ von $ A $}.
\end{definition}
\begin{bemerkung*}
$ K(x_0) $ ist zusammenh\"angend.
\end{bemerkung*}
\begin{definition}
$ (X,d_x) $ metrischer Raum, $ A\subset X $ eine Teilmenge. $ A $ ist \deftxt{wegzusammenh\"angend}$ \Leftrightarrow\forall x_0,x_1\in A \exists$stetige Abbildung $ \gamma\colon[0,1]\rightarrow A $ so dass $ \gamma(0)=x_0 $, $ \gamma(1)=x_1 $.
\end{definition}
\begin{bemerkung*}
$ A\subset X $ wegzusammenh\"angend$ \nLeftarrow\Rightarrow A$ zusammenh\"angend.
\begin{beispiel*}
$ \R^2: y=\sin\frac{1}{x} $, $ 0<x\leq 1 $, $ A=(\lbrace 0\rbrace\times[-1,1])\cup\left\lbrace\left(x,\sin\frac{1}{x}\right),0<x\leq 1\right\rbrace $
\end{beispiel*}
\end{bemerkung*}
\begin{proposition}
$ A\subset\R^2 $ offen, $ d_{\R^2} (x,y)=\norm{x-y}$. $ A $ zusammenh\"angend$ \Rightarrow A $ wegzusammenh\"angend.
\end{proposition}
\begin{beweis}
Sei $ x_0\in A $ beliebig, aber fixiert. $ A(x_0)\coloneqq\lbrace y\in A\mid\exists\gamma\colon[0,1]\rightarrow A, \gamma(0)=x_0,\gamma(1)=x_1\rbrace $.
\begin{enumerate}
\item $ A(x_0) $ ist wegzusammenh\"angend.
\item $ A(x_0) $ ist offen, weil $ \forall y\in A(x_0)\subset A\exists\e>0 $ so dass $ B_\e(y) $. Also ist die Kurve $ \gamma $ von $ x_0 $ zu $ y +$der Radius von $ y $ zu beliebigem Punkt von $ B_\e(y) $ auch eine stetige Kurve. Also ist auch $ B_\e(y)\subset A(x_0) $ und somit ist $ A(x_0) $ offen.
\item $ A(x_0) $ ist abgeschlossen in $ A $. Sei $ y\ast\in A $ und $ \exists y_n\in A(x_0) $, $ y_n\xrightarrow{n\to\infty}y^\ast $. Da $ A $ offen ist, existiert ein $ \e>0 $, so dass $ B_\e(y^\ast)\subset A $. Dann $ \exists n\in\N $ so dass $ y_n\in B_\e(y^\ast) $. Also existiert ein $ \gamma\colon[0,1]\rightarrow A $, so dass $ \gamma(0)=x_0 $, $ \gamma(1)=x_1 $, Dann ist diese Kurve+der Radius $ [y_n,y^\ast] $ eine Kurve die $ x_0 $ mit $ y^\ast $ verbindet. Also $ y^\ast\in A(x_0) $. Somit ist $ A(x_0) $ in $ A $ abgeschlossen.
\end{enumerate}
Also sind $ A(x_0) $ und $ A\setminus A(x_0) $ offen$ \lightning $ $ A\setminus A(x_0)=\emptyset \Rightarrow A(x_0)=A$. Da $ A(x_0) $ wegzusammenh\"angend ist, ist somit auch $ A $ wegzusammenh\"angend.
\end{beweis}
\begin{definition}
	$ f\colon X\rightarrow\C $ hei\ss t \deftxt{lokal-konstant} genau dann wenn $ \forall x\in X\exists $ offene Umgebung $ U\subset X $, $ x\in U $, so dass $ f|_U= $konstant.
\end{definition}
Ist $ f $ lokal-konstant, dann ist $ f $ stetig.\\
\begin{satz}
	$ X $ metrischer Raum. Dann sind \"aquivalent:\begin{enumerate}
		\item $ f\colon X\rightarrow\C $ lokal-konstant$ \Rightarrow f$ konstant
		\item $ A\subset X $ nicht leer, offen und abgeschlossen$ \Rightarrow A=X $
		\item $ X $ zusammenh\"angend
	\end{enumerate}
\end{satz}
\newpage
\begin{beweis}
	\begin{description}
		\item[i)$ \Rightarrow $ii)] Sei $ A\subset X $, $ A\neq\emptyset $, offen und abgeschlossen. $ B\coloneqq X\setminus A $ offen und abgeschlossen, $ A\cap B=\emptyset $, $ f(x)= \begin{cases}
		1&x\in A\\0&x\in B
		\end{cases} $. Es folgt direkt dass $ f $ lokal-konstant, also insbesondere stetig ist. Also ist $ f $ konstant, n\"amlich $ f=1 $, denn $ A\neq\emptyset $. Da $ A=f^{-1}(1)=X $, ist $ A=X $.
		\item[ii)$ \Rightarrow $i)] Sei $ f\colon X\rightarrow\C $ lokal-konstant. Fixiere $ c\in X $. $ A\coloneqq f^{-1}(f(c)) $. Da $ f $ lokal-konstant, ist $ A $ offen, $ c\in A\neq\emptyset $. Da $ f $ stetig, ist $ A $ abgeschlossen. Also ist $ A=X $. Insbesondere ist $ f(x)=f(x)\forall x\in X $. Also ist $ f $ konstant.
	\end{description}
\end{beweis}
\begin{satz}
	$ I\subset\R $ Intervall$ \Rightarrow I $ zusammenh\"angend.
\end{satz}
\begin{definition}
	\bullshit
	\begin{enumerate}
		\item $ z_0,z_1\in\C $, $ \gamma(t)=(1-t)z_0+tz_1 $, $ t\in[0,1] $. $ \gamma $ hei\ss t \deftxt{Strecke} von $ z_0 $ nach $ z_1 $, $ \gamma=[z_0,z_1] $.
		\item $ z_0,z_1\in\R $, dann ist $ [z_0,z_1]= $Intervall.
		\item Seien $ \gamma_1\colon[a_j,b_j]\rightarrow\C $, $ j=1,2 $, $ \gamma_1(b_1)=\gamma_2(a_2) $. Der \deftxt{Summenweg} $ \gamma_1+\gamma_2 $ von $ \gamma_1 $ und $ \gamma_2 $ ist $ \gamma\colon[a_1,b_2-a_2+b_1] $, $ \gamma(t)=\begin{cases}
		\gamma_1(t)&t\in[a_1,b_1]\\\gamma_2(t+a_2-b_1)&t\in[b_1,b_2-a_2+b_1]
		\end{cases} $.
		\item $ \gamma $ hei\ss t \deftxt{Polygon} oder \deftxt{Streckenzug}, falls $ \gamma=[z_0,z_1]+[z_1,z_2]+...+[z_{n-1},z_n] $.
		\item Polygon $ \gamma $ hei\ss t \deftxt{achsenparallel}, falls $ [z_j,z_{j+1}] $ parallel zur $ x- $Achse oder $ y- $Achse ist, $ j=0,...,n-1 $, d.h. $ \Re=z_j=\Re z_{j+1} $ oder $ \Im z_j=\Im z_{j+1} $.
		\item $ D\subset\C $ hei\ss t \deftxt{Bereich}, falls $ D $ offen und nicht leer ist.
	\end{enumerate}
\end{definition}
\begin{satz}
	Sei $ B\subset\C $ Bereich. Dann sind \"aquivalent:
	\begin{enumerate}
		\item $ B $ ist zusammenh\"angend.
		\item $ \forall p,q\in B\exists $Polygon in $ B $, das $ p $ und $ q $ verbindet.
		\item $ B $ ist wegzusammenh\"angend.
	\end{enumerate}
\end{satz}
\newpage
\begin{beweis}
	\begin{description}
		\item[ii)$ \Rightarrow $iii)] Jedes Polygon ist ein Weg.
		\item[iii)$ \Rightarrow $i)] Folgt aus Bemerkung oben.
		\item[i)$ \Rightarrow $ii)] Sei $ p\in B $ fest, $ z\in B $.
		\[ f(z)=\begin{cases}
		1&  \exists \text{Polygon von }  p  \text{ nach }  b \\0& \text{sonst}
		\end{cases} \]
		Zeige: $ f $ lokal konstant. Sei $ w\in B $. Da $ B $ offen, gibt es eine Kreisscheibe $ \triangle\subset B $, $ \triangle\ni w $. Ist $ z\in\triangle $, so existiert ein Polygon von $ z $ nach $ w $ in $ \triangle $. D.h. $ f(w)=1\Rightarrow f(z)=1 $ und $ f(w)=0\Rightarrow f(z)=0 \forall z\in\triangle$. Also ist $ f $ lokal-konstant auf $ B $ und $ f $ somit konstant. Da $ f(p)=1 $ folgt $ f=1 $.
	\end{description}
\end{beweis}
\begin{definition}
	$ G\subset\C $ Bereich. Ist $ G $ (weg-)zusammenh\"angend, so hei\ss t $ G $ \deftxt{Gebiet}.
\end{definition}
$ G $ ab jetzt immer ein Gebiet, und $ D $ immer ein Bereich.\\
\begin{definition}
	$ p,q\in D $ $ p\sim_D q \Leftrightarrow\exists$Weg in $ D $ der $ p $ und $ q $ verbindet. Die \"Aquivalenzklasse $ [p]_D $ hei\ss t \deftxt{Zusammenhangskomponente die $ p $ enth\"alt}.
\end{definition}
$ z_0,z_1\in\C $, $ d(z_0,z_1)=|z_0-z_1| $ Abstand zwischen $ z_0 $ und $ z_1 $.\\
$ z_0\in\C $, $ A\subset\C $ abgeschlossen, $ d(z_0,A)=\inf\lbrace d(z_0,w)\mid w\in A\rbrace $\\
$ D\subset\C $ Bereich, $ c\in D $, $ \partial D=\bar D\setminus D $. Randabstand $ d_c(D)=d(c,\partial D) $.\\
Sonderfall: $ D=\C $, $ d_c(D)=+\infty $.\\
$ d=d_c(D) $ ist der maximale Radius, so dass $ B_d(c)\subset D $ enthalten ist.
\begin{beispiel*}
	$ D=B_r(a) $. $ \partial D=\lbrace z\in\C\mid |z-a|=r\rbrace $.
\end{beispiel*}
\chapter{Topologische Grundbegriffe}
\begin{definition}
Ist $ X $ irgendeine Menge, so hei\ss t eine Funktion $ d\colon X\times X\rightarrow\R $, $ (x,y)\mapsto d(x,y) $, eine \deftxt{Metrik auf $ X $}, wenn $ \forall x,y,z\in X $ gilt:
\begin{enumerate}
\item $ d(x,y)\geq 0 $, $ d(x,y)=0\Leftrightarrow x=y $
\item $ d(x,y)=d(y,x) $
\item $ d(x,z)\leq d(x,y)+d(y,z) $
\end{enumerate}
$ (X,d) $ hei\ss t \deftxt{metrischer Raum}.
\end{definition}Im Fall $ X=\C $ nennt man $ d(w,z)\coloneqq |w-z|=\sqrt{(u-x)^2+(v-y)^2} $ (die \deftxt{euklidische Entfernung} der Punkte $ w,z $ in der Zahlebene) die \deftxt{euklidische Metrik} von $ \C $.\\
In einem metrischen Raum $ X $ mit Metrik $ d $ hei\ss t die Menge \[ B_r(c)\coloneqq\lbrace x\in X\mid d(x,c)<r\rbrace \] die \deftxt{offene Kugel vom Radius $ r>0 $ mit Mittelpunkt $ c\in X $}.\\
Im Fall der euklidischen Metrik auf $ \C $ hei\ss en die Kugeln \[ B_r(c)\coloneqq\lbrace z\in\C\mid |z-c|<r\rbrace \] $ r>0 $, \deftxt{offene Kreisscheibe in $ \C $}. Wir schreiben durchweg \[ \E\coloneqq B_1(0)=\lbrace z\in C\mid |z|<1\rbrace \]
\begin{definition}
Eine Teilmenge $ U\subset X $ eines metrischen Raumes $ X $ hei\ss t \deftxt{offen} (in $ X $)$ \Leftrightarrow\forall x\in U\exists r>0 $ so dass $ B_r(x)\subset U $ ($ \emptyset $ ist offene Menge per definitionem).
\begin{enumerate}
\item $ \lbrace U_\alpha\rbrace_{\alpha\in A}\Rightarrow\bigcup_{\alpha\in A}U_\alpha $ offen
\item $ U_1,U_2,...,U_m $ offen$ \Rightarrow\bigcap_{i=1}^m U_i $ offen
\end{enumerate}
\end{definition}
\newpage
\begin{definition}
Eine Menge $ A\subset X $ hei\ss t \deftxt{abgeschlossen} (in $ X $)$ \Leftrightarrow X\setminus A$ offen.
\begin{enumerate}
\item $ \lbrace A_\alpha\rbrace_{\alpha\in\sA} $ abgeschlossene Mengen$ \Rightarrow\bigcap_{\alpha\in\sA} A_\alpha$ abgeschlossen
\item $ A_1,A_2,...,A_m $ abgeschlossen$ \Rightarrow\bigcup_{i=1}^m A_i $ abgeschlossen
\end{enumerate}
\end{definition}
\begin{definition}
$ A\subset X $ beliebig. Die \deftxt{abgeschlossene H\"ulle $ \bar A $ von $ A $} ist $ \bar A\coloneqq\bigcap B $, so dass $ B\supset A $, $ B $ abgeschlossen. 
\end{definition}
Eine Menge $ W\subset X $ hei\ss t \deftxt{Umgebung der Menge $ M\subset X $}, wenn $ \exists V $ offen mit $ M\subset V\subset W $.\\
Sei $ k\in\N\coloneqq\lbrace 0,1,2,...\rbrace $. Eine Abbildung $ \lbrace k,k+1,k+2,...\rbrace\rightarrow X $, $ n\mapsto c_n $, hei\ss t \deftxt{Folge} in $ X $. Man schreibt kurz $ (c_n) $, im Allgemeinen ist $ k=0 $.\\
\begin{definition}
Eine Folge $ (c_n) $ hei\ss t \deftxt{konvergent} in $ X $, wenn es einen Punkt $ c\in X $ gibt, so dass in jeder Umgebung von $ c $ fast alle (d.h. alle bis auf endlich viele) Folgenglieder $ c_n $ liegen. Der Punkt $ c $ hei\ss t ein \deftxt{Limes der Folge}. In Zeichen:
\[ c=\lim_{n\to\infty}c_n \]
Nicht konvergente Folgen hei\ss en \deftxt{divergent}. 
\end{definition}
Eine Menge $ M\subset X $ ist genau dann abgeschlossen in $ X $, wenn der Limes jeder konvergenten Folge $ (c_n) $, $ c_n\in M $, stets zu $ M $ geh\"ort.\\
\begin{definition}
Ein Punkt $ p\in X $ hei\ss t \deftxt{H\"aufungspunkt} einer Menge $ M\subset X :\Leftrightarrow\forall $Umgebung $ U $ von $ p $ gilt:
\[ U\cap (M\setminus\lbrace p\rbrace)\neq\emptyset \]
\end{definition}
In jeder Umgebung eines H\"aufungspunktes $ p $ von $ M $ liegen unendlich viele Punkte von $ M $; es gibt stets eine Folge $ (c_n) $ in $ M\setminus\lbrace p\rbrace $ mit $ \lim c_n=p $.
\newpage
\begin{beispiel*}
\begin{enumerate}
\item[]
\item $ X=\R $, $ M=\Q $. Die Menge $ U $ aller H\"aufungspunkte? $ U=\R $.
\item $ X=\R $, $ M=\Z $. $ U = \emptyset $.
\item $ X=\R $, $ M=\left\lbrace\frac{1}{n}\right\rbrace_{n=1}^\infty $, $ U=\lbrace 0\rbrace $.
\end{enumerate}
\end{beispiel*}
\begin{definition}
Eine Teilmenge $ A $ eines metrischen Raumes $ X $ hei\ss t \deftxt{dicht}, in $ X:\Leftrightarrow\forall $offene $ U\subset X: U\cap A\neq\emptyset \Leftrightarrow\bar A=X$.
\end{definition}
\begin{beispiel*} $ X=C[a,b] $, $ d(f,g)=\sup_{x\in[a,b]} |f(x)-g(x)| $, $ f,g\in X $, $ A=\sP= $ alle Polynome auf $ [a,b] $. 
\end{beispiel*}
\begin{satz}[\"Aquivalenzsatz]
Folgende Aussagen \"uber einen metrischen Raum $ X $ sind \"aquivalent:
\begin{enumerate}
\item Jede offene \"Uberdeckung $ U=\lbrace U_j\rbrace_{j\in J} $ von $ X $ besitzt eine endliche Teil\"uberdeckung. (Heine-Borel-Eigenschaft)
\item Jede Folge $ (x_n) $ in $ X $ besitzt eine konvergente Teilfolge. (Weierstra\ss-Bolzano-Eigenschaft)
\end{enumerate}
\end{satz}
\begin{definition}
Man nennt $ X $ \deftxt{kompakt}, wenn die Bedingungen i) und ii) aus \ref{satz2.8} erf\"ullt sind. Eine Teilmenge $ K $ von $ X $ hei\ss t \deftxt{kompakt}, oder auch ein \deftxt{Kompaktum} (in $ X $), wenn $ K $ mit der induzierten Metrik ein kompakter Raum ist. 
\end{definition}
\begin{enumerate}
\item[($ \ast $)] Jedes Kompaktum in $ X $ ist abgeschlossen in $ X $. In einem kompakten Raum ist jede abgeschlossene Teilmenge kompakt.
\item[($ \ast\ast $)] Jede offene Menge $ D $ in $ \C $ ist die Vereinigung von abz\"ahlbar unendlich vielen kompakten Teilmengen von $ D $.
\end{enumerate}
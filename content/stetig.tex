\section{Stetige Funktionen}
$ f\colon X\rightarrow Y $, $ f $ hei\ss t \deftxt{Funktion} oder \deftxt{Abbildung}, $ X $ hei\ss t \deftxt{Argumentbereich} und $ Y $ \deftxt{Wertebereich}. Man schreibt auch $ X\ni x\rightarrow f(x)\in Y $.\\
\begin{definition}
Eine Abbildung $ f\colon X\rightarrow Y $ hei\ss t \deftxt{stetig im Punkt $ a\in X $}, wenn das $ f- $Urbild $ f^{-1}(V)=\lbrace x\in X\mid f(x)\in V\rbrace $ einer jeden Umgebung $ V $ von $ f(a) $ in $ Y $ eine Umgebung von $ a $ in $ X $ ist.
\end{definition}
\begin{definition}
Die Funktion $ f\colon X\rightarrow Y $ konvergiert bei Ann\"aherung an $ a\in X $ gegen $ b\in Y $, in Zeichen $ \lim_{x\to a} f(x)=b $ oder $ f(x)\rightarrow b $ wenn $ x\rightarrow a $, wenn es zu jeder Umgebung $ V $ von $ b $ in $ Y $ eine Umgebung $ U $ von $ a $ in $ X $ gibt mit $ f(U\setminus\lbrace a\rbrace)\subset V $.
\end{definition}
\begin{bemerkung*}
$ f $ ist stetig in $ a\Leftrightarrow\exists\lim_{x\to a}f(x)=f(a) $.
\end{bemerkung*}
\begin{satz}[Folgenkriterium]
Genau dann ist $ f\colon X\rightarrow Y $ stetig in $ a $, wenn $ \forall $Folgen $ (x_n) $ von Punkten $ x_n\in X $ mit $ \lim x_n=a $ gilt: $ \lim f(x_n)=f(a) $.
\end{satz}
Zwei Abbildungen $ f\colon X\rightarrow Y $ und $ g\colon Y\rightarrow Z $ werden zusammengesetzt zu $ g\circ f\colon X\rightarrow Z $, $ z\rightarrow (g\circ f)(x)\coloneqq g(f(x)) $. Bei dieser Komposition von Abbildungen vererbt sich die Stetigkeit: Ist $ f\colon X\rightarrow Y $ stetig in $ a\in X $ und ist $ g\colon Y\rightarrow Z $ stetig in $ f(a)\in Y $, so ist $ g\circ f\colon X\rightarrow Z $ stetig in $ a $.\\
\begin{definition}
Eine Funktion $ f\colon X\rightarrow Y $ hei\ss t \deftxt{stetig}, wenn sie in jedem Punkt von $ X $ stetig ist.
\end{definition}
\begin{satz}[Stetigkeitskriterium]
Folgende Aussagen sind \"aquivalent:
\begin{enumerate}
\item $ f $ ist stetig.
\item Das Urbild $ f^{-1}(V) $ jeder in $ Y $ offenen Menge $ V $ ist offen in $ X $.
\item Das Urbild $ f^{-1}(A) $ jeder in $ Y $ abgeschlossenen Menge $ A $ ist abgeschlossen in $ X $.
\end{enumerate}
\end{satz}
\newpage
\begin{satz}
Es sei $ f\colon X\rightarrow Y $ stetig und $ K\subset X $ ein Kompaktum. Dann ist auch $ f(K)\subset Y $ ein Kompaktum.
\end{satz}
\begin{beweis}
Sei $ \lbrace U_\alpha\rbrace_{\alpha\in A} $ eine offene \"Uberdeckung von $ f(K) $. Sei $ W_\alpha\coloneqq f^{-1}(U_\alpha)\forall\alpha\in A $. $ f $ ist stetig, also ist f\"ur alle $ \alpha\in A $ $ W_\alpha $ offen. Also ist $ \lbrace W_\alpha\rbrace_{\alpha\in A} $ eine offene \"Uberdeckung von $ K $. Da $ K $ kompakt ist, existieren endlich viele $ \alpha_1,\alpha_2,...,\alpha_m $, so dass $ K\subset\bigcup_{i=1}^m W_{\alpha_i} $. Dann ist $ \lbrace U_{\alpha_i}\rbrace_{i=1}^m $ eine endliche \"Uberdeckung von $ f(K) $. Somit ist $ f(K) $ nach Definition ein Kompaktum. 
\end{beweis}
In \ref{satz5.6} ist enthalten, dass reellwertige stetige Funktionen $ f\colon X\rightarrow\R $ auf jedem Kompaktum $ K $ in $ X $ Maxima und Minima annehmen.\\
Komplexwertige Funktionen $ f\colon X\rightarrow\C $ und $ g\colon X\rightarrow\C $ lassen sich addieren und multiplizieren: $ (f+g)(x)=f(x)+g(x) $ und $ (f\cdot g)(x)=f(x)g(x) $, $ x\in X $. Die zu $ f $ konjugierte Funktion $ \bar f $ wird durch $ \bar f(x)=\overline{f(x)} $, $ x\in X $, definiert.\\
Rechenregeln: $ \overline{f+g}=\bar f+\bar g $, $ \overline{f\cdot g}=\bar f\cdot\bar g $, $ \bar{\bar f}=f $. Realteil und Imagin\"arteil von $ f $ werden durch $ (\Re f)(x)=\Re (f(x)) $ und $ (\Im f)(x)=\Im (f(x)) $, $ x\in X $, erklärt. F\"ur $ u\coloneqq\Re f $ und $ v\coloneqq \Im f $ (reellwertige Funktionen) gilt: $ f=u+iv $, $ u=\frac{1}{2}(f+\bar f) $, $ v=\frac{1}{2i}(f-\bar f) $, $ f\bar f=u^2+v^2 $.\\
Man hat:
\begin{enumerate}
\item $ f\colon X\rightarrow\C $, $ g\colon X\rightarrow\C $ stetig in $ a\in X \Rightarrow f+g, fg, \bar f$ stetig in $ a $.
\item $ f=u+iv $ stetig in $ a\Leftrightarrow u,v$ stetig in $ a $.
\item $ g $ nullstellenfrei in $ X $ (d.h. $ g(x)\neq 0\forall x\in X $), dann hei\ss t die Funktion $ x\rightarrow \frac{f(x)}{g(x)} $ die \deftxt{Quoiientenfunktion} von $ f $ und $ g $. Sind $ f $ und $ g $ stetig in $ a\Rightarrow\frac{f(x)}{g(x)} $ stetig in $ a $.
\end{enumerate}
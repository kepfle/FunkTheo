\chapter{Integralsatz, Integralformel und Potenzreihenentwicklung}
\section{Cauchyscher Integralsatz f\"ur Sterngebiete}
\begin{lemma}[Integrallemma von Goursat]
Es sei $ f $ holomorph im Bereich $ D $. Dann gilt f\"ur den Rand $ \partial \Delta $ eines jeden Dreiecks $ \Delta\subset D $:
\[ \int_{\partial\Delta}^{} f\dd\zeta = 0 \]
\end{lemma}
\begin{beweis}
Sei $ \int_{\partial\Delta}^{} f\dd\zeta\neq 0 $ und sei \[ \alpha(\Delta)\coloneqq\left|\int_{\partial\Delta}^{} f\dd\zeta\right|\neq 0 \]
Wir teilen $ \Delta $ in vier gleiche Dreiecke $ \Delta_1^1, \Delta_1^2,\Delta_1^3,\Delta_1^4 $. Dann
\[ \int_{\partial\Delta}f\dd\zeta=\sum_{k=1}^{4}\int_{\partial\Delta_1^k}^{}f\dd\zeta \]
Damit existiert ein $ k_1 $, so dass
\[ \left|\int_{\partial\Delta_1^{k_1}}f\dd\zeta\right|\geq\frac{\alpha(\Delta)}{4} \]
Wir teilen $ \Delta_1^{k_1} $ in vier gleiche Dreiecke $ \Delta_2^{k_1,1},\Delta_2^{k_1,2},\Delta_2^{k_1,3},\Delta_2^{k_1,4} $ und bekommen
\[ \int_{\partial\Delta_1^{k_1}}f\dd\zeta=\sum_{k=1}^4\int_{\partial\Delta_2^{k_1,k}}f\dd\zeta \]
Damit existiert ein $ k_2 $, so dass
\[ \left|\int_{\partial\Delta_2^{k_1,k_2}}^{}f\dd\zeta\right|\geq\frac{1}{4}\left|\int_{\partial\Delta_1^k}^{}f\dd\zeta\right|\geq\frac{1}{4^2}\alpha(\Delta) \]
Wir machen genau das gleiche f\"ur $ \Delta_2^{k_!,k_2} $ und bekommen $ \Delta_3^{k_1,k_2,k_3},...,\Delta_m^{k_1,k_2,...,k_m} $, so dass
\[ \left|\int_{\partial\Delta_m^{k_1,...,k_m}}^{} f\dd\zeta\right|\geq\frac{1}{4^m}\alpha(\Delta) \]
Es existiert genau ein \[ p=\bigcap_{m=1}^\infty \Delta_m^{k_1,..,k_m}\subset D\]
$ f\in\sO(D) $, also:
\[ f(\zeta)=f(p)+f'(p)(\zeta-p)+g(\zeta)(\zeta-p),\quad g\in C(D),g(p)=0 \]
Dann:
\[ \int_{\partial\Delta_m^{k_1,...,k_m}}^{}f\dd\zeta=\int_{\partial\Delta_m^{k_1,...,k_m}}^{}f(p)\dd\zeta+\int_{\partial\Delta_m^{k_1,...,k_m}}^{}f'(p)(\zeta-p)\dd\zeta+\int_{\partial\Delta_m^{k_1,...,k_m}}^{}g(\zeta)(\zeta-p)\dd\zeta \]
F\"ur $ f(p) $ ist $ f(p)\zeta $ eine Stammfunktion, f\"ur $ f'(p)(\zeta-p) $ ist $ \frac{1}{2}f'(p)(\zeta-p)^2 $ eine Stammfunktion, also folgt:
\[ \left|\int_{\partial\Delta_m^{k_1,...,k_m}}^{}f\dd\zeta\right|=\left|\int_{\partial\Delta_m^{k_1,...,k_m}}^{}g(\zeta)(\zeta-p)\dd\zeta\right|\leq\sup_{\partial\Delta_m^{k_1,...,k_m}}|g(\zeta)(\zeta-p)|\cdot\frac{l(\Delta)}{2^m}\leq\sup_{\zeta\in\partial\Delta_m^{k_1,...,k_m}}|g(\zeta)|\frac{l(\Delta)^2}{4^m}\xrightarrow{m\to\infty}0 \]
Auf der anderen Seite:
\[ \left|\int_{\partial\Delta_m^{k_1,...,k_m}}f\dd\zeta\right|\geq\frac{1}{4^m}\alpha(\Delta) \]
\[ \frac{1}{4^m}\alpha(\Delta)\geq\sup_{\zeta\in\partial\Delta_m^{k_1,...,k_m}}|g(\zeta)|\frac{l(\Delta)^2}{4^m}\xrightarrow{m\to\infty}0\lightning \]
\end{beweis}
\begin{satz}[Cauchyscher Integralsatz f\"ur Sterngebiete]
Es sei $ G $ ein Sterngebiet mit Zentrum $ c $, es sei $ f\colon G\rightarrow\C $ holomorph in $ G $. Dann ist $ f $ integrabel in $ G $, die Funktion
\[ F(z)\coloneqq\int_{[c,z]}^{} f\dd\zeta,\quad z\in G \]
ist eine Stammfunktion von $ f $ in $ G $. Speziell gilt:
\[ \int_{\gamma}^{} f\dd\zeta=0 \]
f\"ur jeden geschlossenen Weg $ \gamma $ in $ G $.
\end{satz}
\begin{beweis}
Wegen $ f\in\sO(G) $ folgt mit Goursat:
\[ \int_{\partial\Delta}^{} f\dd\zeta=0,\quad\Delta\subset G \]
Mit dem Integrabilit\"atskriterium f\"ur Sterngebiete folgt dann, dass
\[ F(z)=\int_{[c,z]}^{} f\dd\zeta \] eine Stammfunktion von $ f $ ist.
\end{beweis}
\begin{beweis}[Reeller Beweis des Integrallemmas von Goursat:]
Sei $ D\subset\C $ ein Bereich, $ \Sigma\subset D $ mit glattem Rand $ \partial\Sigma $ und $ f\in\sO(D) $.
\begin{align*} \int_{\partial\Sigma} f\dd\zeta&=\int_{\partial\Sigma}^{} (u+iv)(\dd x+i\dd y)\\&=\int_{\partial\Sigma}(u\dd x-v\dd y)+i\int_{\partial\Sigma}^{} (v\dd x+u\dd y)\\&=\iint_\Sigma -\frac{\partial v}{\partial x}-\frac{\partial u}{\partial y}\dd x\dd y+i\iint_\Sigma-\frac{\partial v}{\partial y}+\frac{\partial u}{\partial x}\dd x\dd y\\&=0 \end{align*}
\end{beweis}
\newpage
\section{Cauchysche Integralformel f\"ur Kreisscheiben}
\begin{lemma}[Zentrierungslemma]
Sei $ D\subset\C $ ein Bereich, $ \bar B\subset D $ eine Kreisscheibe, $ B_r(z)\coloneqq\lbrace\eta\mid |z-\zeta|=r\rbrace $ und $ f\in\sO(D\setminus\lbrace z\rbrace) $. Dann ist
\[ \int_{\partial B}^{} f\dd\zeta=\int_{\partial B_r(z)}^{} f\dd\zeta \]
\end{lemma}
\begin{beweis}
Sei $ l $ eine Gerade, so dass $ z\in l $. Wir nehmen $ \Omega_1,\Omega_2 $ wie auf dem Bild (:|). Dann sind $\omega_1\subset\tilde \Omega_1 $ und $ \Omega_2\subset\tilde{\Omega}_2 $ Sterngebiete.
Dann:
\[ \int_{\partial\Omega_1}^{} f\dd\zeta=0,\quad\int_{\partial\Omega_2}^{} f\dd\zeta = 0\Rightarrow\int_{\partial\Omega_1\cup\partial\Omega_2}^{} f\dd\zeta = 0 \]
Es folgt:
\[ \int_{\partial B}^{} f\dd\zeta - \int_{\partial B_r(z)}^{} f\dd\zeta = 0 \]
Die Aussage folgt.
\end{beweis}
\begin{korollar}
	Ist $ g $ beschr\"ankt um $ z $, so gilt:
	\[ \int_{\partial B}g\dd\zeta=0 \]
\end{korollar}
\begin{beweis}
	$ \exists M>0,\e>0 $, so dass $ \forall $Kreis $ S\subset B $ um $ z $ mit Radius $ t<s $ gilt: $ |g|_S\leq M $. Mit dem Zentrierungslemma und der Standardabsch\"atzung haben wir:
	\[ \left|\int_{\partial B}^{} g\dd\zeta\right|=\left|\int_S g\dd\zeta\right|\leq|g|_S2\pi t\leq M2\pi t\forall t>0 \]
	Hieraus folgt die Behauptung.
\end{beweis}
\newpage
\begin{satz}[Cauchysche Integralformel f\"ur Kreisscheiben]
	Es sei $ f $ holomorph im Bereich $ D $, es sei $ B\coloneqq B_r(c) $, $ r>0 $, eine Kreisscheibe, die nebst Rand $ \partial B $ in $ D $ liegt. Dann gilt $ \forall z\in B $:
	\[ f(z)=\frac{1}{2\pi i}\int_{\partial B}^{}\frac{f(\zeta)}{\zeta-z}\dd\zeta \]
\end{satz}
\begin{beweis}
	Sei $ z\in B $ fixiert. Die Funktion $ g(\zeta)=\frac{f(\zeta)-f(z)}{\zeta-z} $ f\"ur $ \zeta\in D\setminus\lbrace z\rbrace $, $ g(z)\coloneqq f'(z) $, ist holomorph in $ D\setminus\lbrace z\rbrace $ und stetig in $ D $. Dann folgt:
	\[ 0=\int_{\partial B}^{}g\dd\zeta=\int_{\partial B}^{}\frac{f(\zeta)}{\zeta-z}\dd\zeta-f(z)\int_{\partial B}^{}\frac{\dd\zeta}{\zeta-z}=\int_{\partial B}^{}\frac{f(\zeta)}{\zeta-z}\dd\zeta-2\pi if(z) \]
	Die Behauptung folgt.
\end{beweis}
\begin{korollar}[Mittelwertgleichung]
	Unter den Voraussetzungen von obigem Satz gilt:
	\[ f(c)=\frac{1}{2\pi}\int_0^{2\pi}f(c+re^{i\theta})\dd\theta \]
\end{korollar}
\begin{beweis}
	\[ f(c)=\frac{1}{2\pi i}\int_0^{2\pi}\frac{f(c+re^{i\theta})}{re^{i\theta}}\dd(c+re^{i\theta})=\frac{1}{2\pi i}\in_0^{2\pi}\frac{f(c+re^{i\theta})rie^{i\theta}{re^{i\theta}}}\dd\theta \]
	Durch K\"urzen erh\"alt man die obige Formel.
\end{beweis}
\begin{korollar}[Mittelwertungleichung]
	\[ |f(c)|\leq |f|_{\partial B_r(c)} \]
\end{korollar}
\section{Entwicklung holomorpher Funktionen in Potenzreihen}
\begin{definition}
	Eine Funktion $ f\colon D\rightarrow\C $ hei\ss t im Kreis $ B=B_r(c)\subset D $ \deftxt{in eine Potenzreihe $ \sum a_\nu(z-c)^\nu $ um $ c $ entwickelbar}, wenn die Potenzreihe in $ B $ gegen $ f|_B $ konvergiert.
\end{definition}
\\
Aus der Vertauschbarkeit von Differentation und Summation f\"ur Potenzreihen folgt sofort:
\begin{satz}
	Ist $ f $ in $ B $ um $ c $ in eine Potenzreihe $ \sum a_\nu(z-c)^\nu $ entwickelbar, so ist $ f $ in $ B $ beliebig oft komplex differenzierbar und es gilt:
	\[ a_\nu=\frac{f^{(\nu)}(c)}{\nu!}\forall \nu\in\N \]
\end{satz}
Eine Potenzreihenentwicklung einer Funktion $ f $ um $ c $ ist also, unabh\"angig vom Radius $ r $ des Kreises $ B $, eindeutig durch die Ableitungen von $ f $ in $ c $ bestimmt und hat immer die Form
\[ f(z)=\sum\frac{f^{(\nu)}(c)}{\nu}(z-c)^\nu \]
Diese Reihe hei\ss t (wie im Reellen) die \deftxt{Taylorreihe von $ f $ um $ c $}. Sie konvergiert in $ B $ normal.\\
\\
Ist $ \gamma $ ein st\"uckweise stetig differenzierbarer Weg in $ \C $, so ordnen wir jeder stetigen Funktion $ f\colon |\gamma|\rightarrow\C $ die Funktion \[ F(z)\coloneqq\frac{1}{2\pi i}\int_\gamma\frac{f(\zeta)}{\zeta-z}\dd\zeta,\quad z\in\C\setminus|\gamma| \]
zu. Wir behaupten:
\begin{lemma}[Entwicklungslemma]
	Die Funktion $ F $ ist in $ \C\setminus|\gamma| $ holomorph. Ist $ c\notin|\gamma| $ irgendein Punkt, so konvergiert die Potenzreihe
	\[ \sum_{0}^{\infty}a_\nu(z-c)^\nu\quad\text{mit}\quad a_\nu\coloneqq\frac{1}{2\pi i}\int_{\gamma}^{}\frac{f(\zeta)}{(\zeta-z)^{\nu+1}}\dd\zeta \]
	in jeder Kreisscheibe um $ c $, die $ |\gamma| $ nicht trifft, gegen $ F $. Die Funktion $ F $ ist beliebig oft differenzierbar in $ \C\setminus|\gamma| $. Es gilt:
	\[ F^{(k)}(z)=\frac{k!}{2\pi i}\int_\gamma\frac{f(\zeta)}{(\zeta-z)^{k+1}}\dd\zeta\forall\zeta\in\C\setminus|\gamma|\forall k\in\N \]
\end{lemma} 
\newpage
\begin{beweis}
	Sei $ B=B_r(c) $ mit $ B\cap|\gamma|=\emptyset $. Die in $ \E $ konvergente Reihe
	\[ \frac{1}{(1-w)^{k+1}}=\sum_{\nu\geq k}^{}\binom{\nu}{k}w^{\nu-k} \]
	liefert (mit $ w\coloneqq\frac{z-c}{\zeta-c} $):
	\begin{align*} \frac{1}{(\zeta-c)^{k+1}}&=\sum_{\nu\geq k}\frac{1}{(\zeta-c)^{\nu+1}}(\zeta-c)^{\nu-k}\forall z\in B,\zeta\in|\gamma|,k\in\N\\
	&=\frac{1}{\left((\zeta-c)-(z-c)\right)^{k+1}}\\
	&=\frac{1}{(\zeta-c)^{k+1}}\frac{1}{\left(1-\left(\frac{z-c}{\zeta-c}\right)\right)^{k+1}}\\
	&=\frac{1}{(\zeta-c)^{k+1}}\sum_{\nu\geq k}^{}\binom{\nu}{k}\left(\frac{z-c}{\zeta-c}\right)^{\nu-k} \end{align*}
	Mit $ g_\nu(\zeta) $, $ \zeta\in|\gamma| $, folgt daher:
	\[ \frac{k!}{2\pi i}\int_{\gamma}^{}\frac{f(\zeta)}{(\zeta-z)^{k+1}}\dd\zeta=\frac{1}{2\pi i}\int_{\gamma}^{}\sum_{\nu\geq k}^{}k!\binom{\nu}{k}g_\nu(\zeta)(z-c)^{\nu-k}\dd\zeta \]
	Da $ |\zeta-c|\geq r\forall\zeta\in|\gamma| $, folgt $ |g_\nu|_{|\gamma|}\leq r^{-(\nu+1)}|f|_{|\gamma|} $ und also
	\[ \max_{\zeta\in|\gamma|}|g_\nu(\zeta)(z-c)^{\nu-k}|\leq\frac{1}{r^{k+1}}|f|_{|\gamma|}q^{\nu-k}\quad\text{mit} \quad q\coloneqq\frac{|z-c|}{r}\]
	Da $ 0\leq q<1\forall z\in B $ und da
	\[ \sum_{\nu\geq k}^{}\binom{\nu}{k}q^{\nu-k}=\frac{1}{(1-q)^{\nu+1}} \]
	konvergiert oben die rechts unter dem Integral stehende Reihe f\"ur feste $ z\in B $ in $ \zeta $ normal auf $ \gamma $. Daher gilt nach dem Vertauschungssatz f\"ur Reihen:
	\[ \frac{k!}{2\pi i}\int_{\gamma}^{}\frac{f(\zeta)}{(\zeta-z)^{k+1}}\dd\zeta=\sum_{\nu\geq k}^{}k!\binom{\nu}{k}a_\nu(z-c)^{\nu-k}\quad\text{mit}\quad a_\nu\coloneqq\frac{1}{2\pi i}\int_{\gamma}^{}\frac{f(\zeta)}{(\zeta-z)^{\nu+1}}\dd\zeta \]
	Damit ist gezeigt, dass die durch oben definierte Funktion $ F $ in der Kreisscheibe $ B $ durch die Potenzreihe $ \sum a_\nu(z-c)^\nu $ dargestellt wird ($ k=0 $), wegen Eigenschafteen von Potenzreihen folgt weiter, dass $ F $ in $ B $ komplex differenzierbar ist und dass gilt:
	\[ F^{(k)}(z)=\sum_{\nu\geq k}^{}k!\binom{\nu}{k}a_\nu(z-c)^{\nu-k},\quad z\in B,k\in\N \]
	Da $ B $ irgendeine Kreisscheibe in $ \C\setminus|\gamma| $, so folgt (2)  und insbesondere $ F\in\sO(\C\setminus|\gamma|) $.
\end{beweis}
\begin{satz}[Entwicklungssatz von Cauchy-Taylor]
	Es sei $ c\in D $, und es sei $ B_d(c) $ die gr\"o\ss te Kreisscheibe um $ c $ in $ D $. Dann ist jede in $ D $ holomorphe Funktion $ f $ um $ c $ in eine Taylorreihe $ \sum a_\nu(z-c)^\nu $ entwickelbar, die in $ B_d(c) $ normal gegen $ f $ konvergiert. Die Taylorkoeffizienten $ a_\nu $ werden gegeben durch die Integrale
	\[ a_\nu=\frac{f^{(\nu)}(c)}{\nu !}=\frac{1}{2\pi i}\int_{\partial B}\frac{f(\zeta)}{(\zeta-c)^{\nu+1}}\dd\zeta\quad(4) \]
	wobei $ B\coloneqq B_r(c) $ mit $ 0<r<d $. 
\end{satz}
Insbesondere ist $ f $ beliebig oft komplex differenzierbar in $ D $. In jeder Kreisscheibe in $ B $ gelten die Cauchyschen Integralformeln
\[ f^{(k)}(z)=\frac{k!}{2\pi i}\int_{\partial B}^{}\frac{f(\zeta)}{(\zeta-z)^{k+1}}\dd\zeta, z\in B,\forall k\in\N\quad (5) \]
\begin{beweis}
	Wegen $ f\in\sO(D) $ gilt f\"ur jeden Kreis $ B=B_r(c) $, $ 0<r<d $, die Cauchysche Formel
	\[ f(z)=\frac{1}{2\pi i}\int_{\partial B}^{}\frac{f(\zeta)}{\zeta-z}\dd\zeta,z\in B \]
	Nach dem Entwicklungslemma (mit $ F\coloneqq f $, $ \gamma=\partial B $) hat $ f $ also in $ c $ eine in $ B_r(c) $ konvergente Taylorentwicklung mit den durch $ (4) $ gegebenen Taylorkoeffizienten. jede Wahl von $ r<d $ f\"uhrt zur gleichen Reihe. Insbesondere herrscht Konvergenz gegen $ f $ in $ B_d(c) $. Die Identit\"aten $ (5) $ folgen ebenfalls direkt aus dem Entwicklungslemma.
\end{beweis}
\\
F\"ur jede Menge $ A\subset D $ zeigt man die \"Aquivalenz folgender Aussagen:
\begin{enumerate}
	\item Jeder Punkt von $ A $ hat eine Umgebung $ U $, so dass $ U\cap A $ endlich ist.
	\item $ A $ ist abgeschlossen in $ D $, und jeder Punkt $ p\in A $ ist ein isolierter Punkt von $ A $ (d.h. hat eine Umgebung $ U $ mit $ U\cap A=\lbrace p\rbrace $).
	\item F\"ur jedes Kompaktum $ K\subset D $ ist $ K\cap A $ endlich.
\end{enumerate}
\begin{definition}
	Mengen, die i)-iii) erf\"ullen hei\ss en \deftxt{lokal endlich in $ D $}.
\end{definition}
Endliche Mengen sind lokal endlich.
\newpage
\begin{definition}
	Ist $ A\subset D $ abgeschlossen und $ f\in\sO(D\setminus A) $, so hei\ss t $ f $ \deftxt{stetig} bzw. \deftxt{holomorph nach $ A $ fortsetzbar}, wenn es eine in $ D $ stetige bzw. holomorphe Funktion $ \tilde f\colon D\rightarrow\C $ gibt, so dass $ \tilde f|_{D\setminus A}=f|_{D\setminus A} $.
\end{definition}
\begin{satz}[Riemannscher Fortsetzungssatz]
	Ist $ f $ lokal endlich in $ D $, so sind folgende Aussagen \"uber eine in $ D\setminus A $ holomorphe Funktion \"aquivalent:
	\begin{enumerate}
		\item $ f $ ist holomorph nach $ A $ fortsetzbar.
		\item $ f $ ist stetig nach $ A $ fortsetzbar.
		\item $ f $ ist in einer Umgebung $ U\subset D $ eines jeden Punktes $ c\in A $ beschr\"ankt.
		\item \[ \lim_{z\to c}(z-c)f(z)=0\forall c\in A \]
	\end{enumerate}
\end{satz}
\begin{beweis}
	i)$ \Rightarrow $ii)$ \Rightarrow $iii)$ \Rightarrow $iv) ist trivial. Wir zeigen iv)$ \Rightarrow $i).\\
	Wir nehmen $ C=0 $ an: Wir betrachten die Funktionen
	\[ g(z)=zf(z)\forall z\in D\setminus\lbrace 0\rbrace,\quad g(0)\coloneqq 0,\quad h(z)\coloneqq zg(z) \]
	$ g\in C(D) $ wegen iv). Dann folgt $ h(z)=h(0)+zg(z) $ ist im Nullpunkt komplex differenzierbar mit $ h'(0)=g(0)=0 $. Aus $ h\in\sO(D\setminus\lbrace 0\rbrace) $ folgt $ h\in\sO(D) $ und mit dem Entwicklungslemma:
	\[ h(z)=a_0+a_1z+a_2z^2+... \]
	Wegen $ h(0)=h'(0)=0 $ folgt
	\[ h(z)=z^2(a_2+a_3z+a_4z^2+...) \]
	Da $ h(z)=z^2f(z) $ f\"ur $ z\in D\setminus\lbrace 0\rbrace $ ist
	\[ \tilde f(z)=a_2+a_3z+a_4z^2+... \]
\end{beweis}
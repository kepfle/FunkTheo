\chapter{Laurentreihen und Fourierreihen}
\[ A=A_{r,s}(c)\coloneqq\lbrace z\in\C\mid 0\leq r<|z-c|<s\leq\infty\rbrace \] ist ein \deftxt{Kreisring} um $ c $ mit innerem Radius $ r $ und \"ausserem Radius $ s $. $ A=A^+\cap A^- $ mit $ A^+\coloneqq B_s(c) $, $ A^-\coloneqq\C\setminus\bar B_r(c) $.
\begin{satz}
	Es sei $ f\in\sO(A_{r,s}(c)) $. Dann gilt:
	\[ \int_{S_\rho} f\dd\zeta=\int_{S_\sigma} f\dd\zeta\forall \rho,\sigma\in\R\text{ mit }r<\rho\leq\sigma<s, S_\rho\coloneqq\lbrace z\in\C\mid |z-c|=\rho\rbrace \]
\end{satz}
\begin{beweis}
	Sei $ \gamma\coloneqq S_\sigma-I-S_\rho+I $. Dann ist $ \gamma\sim 0 $, d.h. $ B_\sigma(c)\setminus(\overline{B_\rho(c)}\cup I) $ ist einfach zusammenh\"angend. Also:
	\[ \int_{\gamma}^{}f\dd\zeta=0 \]
	und
	\[ \int_{S_\rho} f\dd\zeta-\int_I f\dd\zeta-\int_{S_\rho}f\dd\zeta+\int_I f\dd\zeta=0 \]
\end{beweis}
\begin{satz}[Cauchscher Integralsatz f\"ur Kreisringe]
	$ f\in\sO(D) $, $ A=A^+\cap A^- $ ein Kreisring um $ c\in D $ so dass $ \bar A\subset D $. Dann gilt:
	\[ f(z)=\frac{1}{2\pi i}\int_{\partial A}^{}\frac{f(\zeta)}{\zeta-z}\dd\zeta=\frac{1}{2\pi i}\int_{\partial A^+}^{}\frac{f(\zeta)}{\zeta-z}\dd\zeta-\frac{1}{2\pi i}\int_{\partial A^-}^{}\frac{f(\zeta)}{\zeta-z}\dd\zeta\forall z\in A \]
\end{satz}
\newpage
\begin{beweis}
	Folgt direkt aus der allgemeinen Version des Cauchyschen Satzes.
\end{beweis}
\section{Laurentdarstellung in Kreisringen}
\begin{definition}
	Ist $ h $ eine komplexe Funktion in einem unbeschr\"ankten Bereich $ W $, so schreiben wir $ \lim_{z\to\infty}h(z)=b $, wenn es zu jeder Umgebung $ V $ von $ b\in\C $ ein $ R>0 $ gibt, so dass $ h(z)\in V\forall z\in W $ mit $ |z|\leq R $
\end{definition}
\begin{satz}
	Es sei $ f\in\sO(\bar A) $, $ A=A^+\cap A^- $ ein Kreisring um $ c $ mit Radien $ r,s $. Dann existieren $ f^+\in\sO(A^+) $ und $ f^-\in\sO(A^-) $ so dass gilt: $ f=f^++f^- $ in $ A $ und $ \lim_{z\to\infty} f^-(z)=0 $. Die Funktionen $ f^+ $ und $ f^- $ sind hierdurch eindeutig bestimmt. F\"ur jedes $ \rho\in[r,s] $ gilt:
	\[ f^+(z)=\frac{1}{2\pi i}\int_{S_\rho}\frac{f(\zeta)}{\zeta-z}\dd\zeta,z\in B_\rho(c) \]
	\[ f^-(z)=\frac{-1}{2\pi i}\int_{S_\rho}\frac{f(\zeta)}{\zeta-z}\dd\zeta,z\in\C\setminus\overline{B_\rho(c)} \]
\end{satz}
\begin{beweis}
	\begin{description}
		\item[Existenz:] Die Funktion \[ f^+_\rho(z)=\frac{1}{2\pi i}\int_{S_{\rho}}\frac{f(\zeta)}{\zeta-z}\dd\zeta, z\in B_\rho(c) \] ist holomorph in $ B_\rho(c) $. F\"ur $ \sigma\in(\rho, s) $ gilt: $ f^+_{\rho}=f^+_\sigma|_{B_\rho(c)} $ nach dem Integralsatz. Es gibt also eine Funktion $ f^+\in\sO(A^+) $ die in $ B_\rho(c) $	mit $ f^+ $ \"ubereinstimmt. Ebenso ist
		\[ f^-(z)\coloneqq f^-_\sigma(z)\coloneqq\frac{-1}{2\pi i}\int_{S_\sigma}\frac{f(\zeta)}{\zeta-z}\dd\zeta,z\in A^-, r<\sigma<\min\lbrace s,|z-c|\rbrace \]
		holomorph in $ A^- $. Die Integralformel, angewendet auf alle Kreisringe $ A' $ um $ c $ mit $ \bar A'\subset A $, liefert in $ A $ die Darstellung $ f=f^++f^- $. Die Standardabsch\"atzung f\"ur Integrale gilt f\"ur $ z\in A^- $:
		\[ |f^-(z)|\leq\sigma\max_{\zeta\in S_\sigma}|f(\zeta)(\zeta-z)^{-1}|\leq\frac{\sigma}{|z-c|-\sigma}|f|_{S_\sigma} \]
		also $ \lim_{z\to\infty}f^-(z)=0 $.
		\item[Eindeutigkeit:] Es seien $ g^+\in\sO(A^+) $, $ g^-\in\sO(A^-) $ weitere Funktionen mit $ f=g^++g^- $ in $ A $ und $ \lim_{z\to\infty} g^-(z)=0 $. Dann gilt:
		\[ f^+-g^+=g^--f^- \]
		auf $ A $. Daher wird durch $ h\coloneqq f^+-g^+ $ auf $ A^+ $ und $ h\coloneqq g^--f^- $ auf $ A^- $ eine ganze Funktion $ h\colon\C\rightarrow\C $ mit $ \lim_{z\to\infty}h(z)=0 $ definiert. $ h $ ist beschr\"ankt auf $ \C $ und mit Liouville ist $ h(z)\equiv $const. Wegen dem Limes ist $ h(z)\equiv 0 $, also $ g^+\equiv f^+ $ und $ g^-\equiv f^- $.
	\end{description}
\end{beweis}
Man nennt die Darstellung von $ f $ als Summe $ f^++f^- $ die \deftxt{Laurentdarstellung von $ f $ in $ A $}. Die Funktion $ f^- $ hei\ss t der \deftxt{Hauptteil}, und $ f^+ $ \deftxt{Nebenteil} von $ f $.
\begin{definition}
	Reihen der Form
	\[ \sum_{-\infty}^{\infty}a_\nu(z-c)^\nu \]
	hei\ss en \deftxt{Laurentreihen um $ c $}. Die Reihen
	\[ \sum_{-\infty}^{-1}a_\nu(z-c)^\nu=\sum_{1}^{\infty}a_{-\nu}(z-c)^{-\nu}\quad\text{bzw.}\quad\sum_0^\infty a_\nu(z-c)^\nu \]
	hei\ss en \deftxt{Hauptteil} bzw. \deftxt{Nebenteil}.
\end{definition}
\begin{satz}[Entwicklungssatz von Laurent]
	Jede im Kreisring $ A $ um $ c $ mit den Radien $ r $ und $ s $ holomorphe Funktion $ f $ ist in $ A $ eindeutig in eine Laurentreihe
	\[ f(z)=\sum_{-\infty}^{\infty}a_\nu(z-c)^\nu \]
	entwickelbar, die in $ A $ normal gegen $ f $ konvergiert. Es gilt:
	\[ a_\nu\coloneqq\frac{1}{2\pi i}\int_{S_\rho}\frac{f(\zeta)}{(\zeta-c)^{\nu+1}}\dd\zeta\qquad(\ast) \]
	f\"ur $ r<\rho<s $, $ \nu\in\Z $.
\end{satz}
\begin{beweis}
	Sei $ f=f^++f^- $ die Laurentdarstellung von $ f $ in $ A=A^+\cap A^- $. Dann hat der Nebenteil $ f^+\in\sO(A^+) $ von $ f $ nach dem Satz von Cauchy-Taylor in $ A^+=B_s(c) $ eine Taylorentwicklung.
	\[ f^+(z)=\sum_{0}^{\infty}a_\nu(z-c)^\nu \]
	Doch auch der Hauptteil $ f^-\in\sO(A^-) $ von $ f $ gestattet eine einfache Reihenentwicklung in $ A^-=\lbrace z\in\C\mid |z-c|>r\rbrace $: Da die Abbildung $ B_{r^{-1}}(0)\setminus 0\rightarrow A^- $, $ w\mapsto z\coloneqq c+w^{-1} $ biholomorph ist mit $ z\mapsto w=(z-c)^{-1} $ als Umkehrabbildung, gilt $ g(w)\coloneqq f^-(c+w^{-1})\in\sO(B_{r^{-1}}(0)\setminus 0) $. Wir wissen, dass $ \lim_{z\to\infty}f^-(z)=0 $, also $ \lim_{w\to 0}g(w)=0 $. Mit dem Riemannschen Fortsetzungssatz folgt dann $ g\in\sO(B_{r^{-1}}(0)) $. Also ist
	\[ g(w)=\sum_{\nu=1}^\infty b_\nu w^\nu \]
	und
	\[ f^-(z)=\sum_{n=1}^\infty b_\nu(z-c)^{-\nu} \]
	Wir setzen $ a_\nu=b_{-\nu} $ f\"ur $ \nu=-1,-2,... $ und bekommen
	\[ f(z)=f^+(z)+f^-(z)=\sum_{\nu=0}^{\infty}a_\nu(z-c)^\nu+\sum_{-\infty}^{-1}a_\nu(z-c)^\nu=\sum_{-\infty}^{\infty}a_\nu(z-c)^\nu \]
	Diese Reihe konvergiert normal in $ A $.
	\[ (z-c)^{-n-1}f(z)=\sum_{-\infty}^{-1}a_{\nu+n+1}(z-c)^\nu+\sum_{\nu=0}^\infty a_{\nu+n+1}(z-c)^\nu \]
	\begin{align*} \int_{S_\rho}(z-c)^{-n-1}f(z)\dd z&=\int_{S_\rho}\left(\sum_{-\infty}^{-1}a_{\nu+n+1}(z-c)^\nu+\sum_{\nu=0}^\infty a_{\nu+n+1}(z-c)^\nu\right)\dd z\\&=\sum_{\nu=-\infty}^{-1}a_{\nu+n+1}\int_{S_\rho}(z-c)^{\nu}\dd z+\sum_{\nu=0}^{\infty}\int_{S_\rho}a_{\nu+n+1}(z-c)^{\nu}\dd z\\&=2\pi i a_n \end{align*}
	Hieraus folgt $ (\ast) $
\end{beweis}
\section{Eigenschaften von Laurentreihen}
\begin{satz}[Identit\"atssatz]
	Sind $ \sum_{-\infty}^{\infty}a_\nu(z-c)^\nu $ und $ \sum_{-\infty}^{\infty}b_\nu(z-c)^\nu $ Laurentreihen, die beide auf einer Kreislinie $ S_\rho $, $ \rho>0 $, um $ c $ gleichm\"a\ss ig konvergieren gegen dieselbe Grenzfunktion $ f $, so gilt:
	\[ a_\nu=b_\nu=\frac{1}{2\pi\rho^\nu}\int_0^{2\pi}f(c+\rho e^{i\varphi})e^{-i\nu\varphi}\dd\varphi, \nu\in\Z \] 
\end{satz}
\begin{beweis}
	Folgt direkt aus $ (\ast) $.
\end{beweis}
\begin{satz}[Gutzmersche Formel und Cauchysche Ungleichungen]
	Konvergiert die Laurentreihe $ \sum_{-\infty}^{\infty}a_\nu(z-c)^\nu $ auf der Kreislinie $ S_\rho $ um $ c $ gleichm\"a\ss ig gegen $ f\colon S_\rho\rightarrow\C $, so gilt die Gutzmersche Formel:
	\[ \sum_{-\infty}^{\infty}|a_\nu|^2\rho^{2\nu}=\frac{1}{2\pi}\int_0^{2\pi}|f(c+\rho e^{i\varphi})|^2\dd\varphi\leq M(\rho)^2\qquad (\ast) \]
	mit $ M(\rho)\coloneqq |f|_{S_\rho} $, insbesondere bestehen die Cauchyschen Ungleichungen
	\[ |a_\nu|\leq\frac{M(\rho)}{\rho^\nu}\forall\nu\in\Z\qquad(\ast\ast) \]
\end{satz}
\begin{beweis}
	Wegen
	\[ \overline{f(c+\rho e^{i\varphi})}=\sum_{-\infty}^{\infty}\bar a_\nu\rho^\nu e^{-i\nu\varphi} \]
	folgt direkt:
	\[ |f(c+\rho e^{i\varphi})|^2=\sum_{-\infty}^{\infty}\bar a_\nu\rho^\nu f(c+\rho e^{i\varphi})e^{-i\nu\varphi} \]
	Dies konvergiert normal auf $ [0,2\pi] $, also:
	\[ \int_0^{2\pi}|f(c+\rho e^{i\varphi})|^2\dd\varphi=\sum_{-\infty}^{\infty}\bar a_\nu\rho^\nu\int_0^{2\pi} f(c+\rho e^{i\varphi})e^{-i\nu\varphi}\dd\varphi=2\pi\sum_{-\infty}^{\infty}|a_\nu|^2\rho^{2\nu} \]
	Der Rest von $ (\ast) $ folgt mit der Standardabsch\"atzung.
	$ (\ast\ast) $ ist trivial.
\end{beweis}
\begin{satz}[Klassifizierung isolierter Singularit\"aten]
	Es sei $ c\in D $ eine isolierte Singularit\"at von $ f\in\sO(D\setminus c) $ und es sei $ f(z)=\sum_{-\infty}^{\infty}a_\nu(z-c)^\nu $ die Laurententwicklung von $ f $ um $ c $. Dann ist $ c $
	\begin{enumerate}
		\item eine hebbare Singularit\"at$ \Leftrightarrow a_\nu=0\forall\nu < 0 $.
		\item ein Pol der Ordnung $ m\geq 1\Leftrightarrow a_\nu=0\forall \nu<-m $ und $ a_{-m}\neq 0 $.
		\item eine wesentliche Singularit\"at$ \Leftrightarrow a_\nu\neq 0$ f\"ur unendlich viele $ \nu <0 $.
	\end{enumerate}
\end{satz}
\chapter{Laurentreihen und Fourierreihen}
\[ A=A_{r,s}(c)\coloneqq\lbrace z\in\C\mid 0\leq r<|z-c|<s\leq\infty\rbrace \] ist ein \deftxt{Kreisring} um $ c $ mit innerem Radius $ r $ und \"ausserem Radius $ s $. $ A=A^+\cap A^- $ mit $ A^+\coloneqq B_s(c) $, $ A^-\coloneqq\C\setminus\bar B_r(c) $.
\begin{satz}
	Es sei $ f\in\sO(A_{r,s}(c)) $. Dann gilt:
	\[ \int_{S_\rho} f\dd\zeta=\int_{S_\sigma} f\dd\zeta\forall \rho,\sigma\in\R\text{ mit }r<\rho\leq\sigma<s, S_\rho\coloneqq\lbrace z\in\C\mid |z-c|=\rho\rbrace \]
\end{satz}
\begin{beweis}
	Sei $ \gamma\coloneqq S_\sigma-I-S_\rho+I $. Dann ist $ \gamma\sim 0 $, d.h. $ B_\sigma(c)\setminus(\overline{B_\rho(c)}\cup I) $ ist einfach zusammenh\"angend. Also:
	\[ \int_{\gamma}^{}f\dd\zeta=0 \]
	und
	\[ \int_{S_\rho} f\dd\zeta-\int_I f\dd\zeta-\int_{S_\rho}f\dd\zeta+\int_I f\dd\zeta=0 \]
\end{beweis}
\begin{satz}[Cauchscher Integralsatz f\"ur Kreisringe]
	$ f\in\sO(D) $, $ A=A^+\cap A^- $ ein Kreisring um $ c\in D $ so dass $ \bar A\subset D $. Dann gilt:
	\[ f(z)=\frac{1}{2\pi i}\int_{\partial A}^{}\frac{f(\zeta)}{\zeta-z}\dd\zeta=\frac{1}{2\pi i}\int_{\partial A^+}^{}\frac{f(\zeta)}{\zeta-z}\dd\zeta-\frac{1}{2\pi i}\int_{\partial A^-}^{}\frac{f(\zeta)}{\zeta-z}\dd\zeta\forall z\in A \]
\end{satz}
\newpage
\begin{beweis}
	Folgt direkt aus der allgemeinen Version des Cauchyschen Satzes.
\end{beweis}
\section{Laurentdarstellung in Kreisringen}
\begin{definition}
	Ist $ h $ eine komplexe Funktion in einem unbeschr\"ankten Bereich $ W $, so schreiben wir $ \lim_{z\to\infty}h(z)=b $, wenn es zu jeder Umgebung $ V $ von $ b\in\C $ ein $ R>0 $ gibt, so dass $ h(z)\in V\forall z\in W $ mit $ |z|\leq R $
\end{definition}
\begin{satz}
	Es sei $ f\in\sO(\bar A) $, $ A=A^+\cap A^- $ ein Kreisring um $ c $ mit Radien $ r,s $. Dann existieren $ f^+\in\sO(A^+) $ und $ f^-\in\sO(A^-) $ so dass gilt: $ f=f^++f^- $ in $ A $ und $ \lim_{z\to\infty} f^-(z)=0 $. Die Funktionen $ f^+ $ und $ f^- $ sind hierdurch eindeutig bestimmt. F\"ur jedes $ \rho\in[r,s] $ gilt:
	\[ f^+(z)=\frac{1}{2\pi i}\int_{S_\rho}\frac{f(\zeta)}{\zeta-z}\dd\zeta,z\in B_\rho(c) \]
	\[ f^-(z)=\frac{-1}{2\pi i}\int_{S_\rho}\frac{f(\zeta)}{\zeta-z}\dd\zeta,z\in\C\setminus\overline{B_\rho(c)} \]
\end{satz}
\begin{beweis}
	\begin{description}
		\item[Existenz:] Die Funktion \[ f^+_\rho(z)=\frac{1}{2\pi i}\int_{S_{\rho}}\frac{f(\zeta)}{\zeta-z}\dd\zeta, z\in B\rho(c) \] ist holomorph in $ B_\rho(c) $. F\"ur $ \sigma\in(\rho, s) $ gilt: $ f^+_{\rho}=f^+_\sigma|_{B_\rho(c)} $ nach dem Integralsatz. Es gibt also eine Funktion $ f^+\in\sO(A^+) $ die in $ B_\rho(c) $	mit $ f^+ $ \"ubereinstimmt. Ebenso ist
		\[ f^-(z)\coloneqq f^-_\sigma(z)\coloneqq\frac{-1}{2\pi i}\int_{S_\sigma}\frac{f(\zeta)}{\zeta-z}\dd\zeta,z\in A^-, r<\sigma<\min\lbrace s,|z-c|\rbrace \]
		holomorph in $ A^- $. Die Integralformel, angewendet auf alle Kreisringe $ A' $ um $ c $ mit $ \bar A'\subset A $, liefert in $ A $ die Darstellung $ f=f^++f^- $. Die Standardabsch\"atzung f\"ur Integrale gilt f\"ur $ z\in A^- $:
		\[ |f^-(z)|\leq\sigma\max_{\zeta\in S_\sigma}|f(\zeta)(\zeta-z)^{-1}|\leq\frac{\sigma}{|z-c|-\sigma}|f|_{S_\sigma} \]
		also $ \lim_{z\to\infty}f^-(z)=0 $.
		\item[Eindeutigkeit:] Es seien $ g^+\in\sO(A^+) $, $ g^-\in\sO(A^-) $ weitere Funktionen mit $ f=g^++g^- $ in $ A $ und $ \lim_{z\to\infty} g^-(z)=0 $. Dann gilt:
		\[ f^+-g^+=g^--f^- \]
		auf $ A $. Daher wird durch $ h\coloneqq f^+-g^+ $ auf $ A^+ $ und $ h\coloneqq g^--f^- $ auf $ A^- $ eine ganze Funktion $ h\colon\C\rightarrow\C $ mit $ \lim_{z\to\infty}h(z)=0 $ definiert. $ h $ ist beschr\"ankt auf $ \C $ und mit Liouville ist $ h(z)\equiv $const. Wegen dem Limes ist $ h(z)\equiv 0 $, also $ g^+\equiv f^+ $ und $ g^-\equiv f^- $.
	\end{description}
\end{beweis}
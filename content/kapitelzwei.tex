\chapter{Konvergenzbegriffe der Funktionentheorie}
\section{Gleichm\"a\ss ige, lokal-gleichm\"a\ss ige und kompakte Konvergenz}
\begin{definition}
	Eine Funktionenfolge $ f_n\colon X\rightarrow\C $ hei\ss t in $ A\subset\C $ \deftxt{gleichm\"a\ss ig konvergent gegen $ f\colon A\rightarrow\C $}, wenn $ \forall\e>0\exists n_0(\e)\in\N $ so dass $ \forall n\geq n_0 $ und $ \forall x\in X $ gilt:
	\[ |f(x)-f_n(x)|<\e \]
\end{definition}
\begin{beispiel*}
	\begin{enumerate}
		\item 	$ f_n(x)=x^n $, $ A=[0,1) $. $ f(x)\equiv 0\forall x\in A $. $ f_n\xrightarrow{n\to\infty}f $ punktweise, aber nicht gleichm\"a\ss ig.
		\item $ A=[0,1] $.
		\[ f_n(z)=\begin{cases}
		2nx&f\in[0,\sfrac{1}{2n}]\\
		2-2nx&f\in(\sfrac{1}{2n},\sfrac{1}{n}]\\
		0&x\in(\sfrac{1}{n},1]
		\end{cases} \]
		$ f(x)\equiv 0\forall x\in[0,1] $.
	\end{enumerate}
\end{beispiel*}
%
%
%
%
%
%Vorlesung nachtragen
%
%
%
%
%
%
